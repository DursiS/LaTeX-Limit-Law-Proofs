\documentclass{article}
\usepackage{amsmath, amssymb, amsthm, sectsty}
\begin{document}
	
\textbf{Constant Multiple Law}
	
\section*{Claim: $\lim_{x \to a} \left( a \cdot f(x) \right) = a \cdot L$}


\begin{proof}


Using the $\varepsilon$–$\delta$ definition of a limit, we will directly prove that:
\[
\lim_{x \to a} af(x) = a \cdot L
\]

\bigskip

The definition tell us, that for every $\varepsilon > 0$, there exists a $\delta > 0$ such that if $0 < |x - a| < \delta$, then $|f(x) - L| < \varepsilon$.

\bigskip

Suppose $0 < |x - a| < \delta$. Let us analyze the distance from the limit point:

\[
|af(x) - aL| = |a||f(x) - L|.
\]


We're looking for when:
\[
|af(x) - aL| < \varepsilon.
\]

From scratch work, this will happen if:
\[
|f(x) - L| < \frac{\varepsilon}{|a|}.
\]

\bigskip

So we choose:
\[
\delta = \frac{\varepsilon}{|a|}.
\]

\bigskip

Then, for every $\varepsilon > 0$, there exists a $\delta > 0$ (namely, $\delta = \varepsilon/|a|$) such that if $0 < |x - a| < \delta$, then
\[
|f(x) - L| < \frac{\varepsilon}{|a|} \quad \Rightarrow \quad |af(x) - aL| < \varepsilon.
\]

\bigskip

Hence, we can conclude from the definition of a limit:
\[
\lim_{x \to a} af(x) = a \cdot L.
\]





\end{proof}
\end{document}