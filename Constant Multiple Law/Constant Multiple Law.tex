\documentclass{article}
\usepackage{amsmath, amssymb, amsthm, sectsty}
\begin{document}
	
\textbf{Constant Multiple Law}
	
\section*{Claim: $\lim_{x \to a} \left( a \cdot f(x) \right) = a \cdot \lim_{x \to a} f(x)$}


\begin{proof}


We will directly prove that $\lim_{x \to a} af(x) = a \cdot L$ using the $\varepsilon$--$\delta$ definition.

\bigskip

Let us recall the $\varepsilon$--$\delta$ definition:  
For every $\varepsilon > 0$, there exists $\delta > 0$ such that if $0 < |x - a| < \delta$, then $|f(x) - L| < \varepsilon$.

\bigskip

Then in our case, given $\varepsilon > 0$, we want to find a $\delta > 0$ such that if $0 < |x - a| < \delta$, then $|af(x) - aL| < \varepsilon$.

\bigskip

Now suppose $0 < |x - a| < \delta$. Let us analyze $|af(x) - aL|$:

\[
|af(x) - aL| = |a||f(x) - L| < \varepsilon
\]

\bigskip

To guarantee this, we need:

\[
|f(x) - L| < \frac{\varepsilon}{|a|}
\]

\bigskip

So, let us define $\delta = \frac{\varepsilon}{|a|}$. Then, whenever $0 < |x - a| < \delta$, it follows that:

\[
|f(x) - L| < \frac{\varepsilon}{|a|} \Rightarrow |a||f(x) - L| < \varepsilon \Rightarrow |af(x) - aL| < \varepsilon
\]

\bigskip

Thus, for every $\varepsilon > 0$, we have found a $\delta$ such that if $0 < |x - a| < \delta$, then $|af(x) - aL| < \varepsilon$.

\bigskip

Hence, $\lim_{x \to a} (a \cdot f(x)) = a \cdot L$.





\end{proof}
\end{document}