\documentclass{article}
\usepackage{amsmath, amssymb, amsthm, sectsty}
\begin{document}
	
\textbf{Sum Law}

\section*{Claim: $\lim_{x \to a} [f(x) + g(x)] = M + N$}


\begin{proof}


Using the $\varepsilon$–$\delta$ definition of a limit, we will directly prove that:
\[
\lim_{x \to a} [f(x) + g(x)] = M + N
\]

The definition tell us, that for every $\varepsilon > 0$, there exists a $\delta > 0$ such that if $0 < |x - a| < \delta$, then $|f(x) - L| < \varepsilon$.

Suppose $0 < |x - a| < \delta$. Let us analyze the distance from the limit point:
\[
| (f(x) + g(x)) - (M + N) | \leq |f(x) - M| + |g(x) - N|
\]

We're look for when:
\[
|f(x) - M| < \delta1
\]
\[
|g(x) - N| < \delta2
\]

Now choose $\delta = \min(\delta_1, \delta_2)$:
\[
\begin{aligned}
| (f(x) + g(x)) - (M + N) | 
&= | (f(x) - M) + (g(x) - N) | \\
&\leq |f(x) - M| + |g(x) - N| \\
&< \frac{\varepsilon}{2} + \frac{\varepsilon}{2} = \varepsilon
\end{aligned}
\]

Hence, for every $\varepsilon > 0$, there exists a $\delta > 0$ such that if $0 < |x - c| < \delta$, then 
$|f(x) - L| < \varepsilon$. Thus, we can conclude:
\[
\lim_{x \to c} [f(x) + g(x)] = M + N
\]





\end{proof}
\end{document}