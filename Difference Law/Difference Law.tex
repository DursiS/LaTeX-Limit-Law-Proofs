\documentclass{article}
\usepackage{amsmath, amssymb, amsthm, sectsty}

\begin{document}
	
\textbf{Sum Law}

\section*{Claim: $\lim_{x \to a} [f(x) - g(x)] = M - N$}


\begin{proof}

Using the $\varepsilon$–$\delta$ definition of a limit, we will directly prove that:
\[
\lim_{x \to a} [f(x) - g(x)] = M - N
\]

The definition tell us, that for every $\varepsilon > 0$, there exists a $\delta > 0$ such that if $0 < |x - a| < \delta$, then $|f(x) - L| < \varepsilon$.

Suppose $0 < |x - a| < \delta$. Let us analyze the distance between the function and its limit value:
\[
| (f(x) - g(x)) - (M - N) | = | (f(x) - M) - (g(x) - N) |
\]

Using the triangle inequality, we estimate the difference as follows:
\[
|(f(x) - M) - (g(x) - N)| \leq |f(x) - M| + |g(x) - N| < \varepsilon
\]


Since the whole, and the sum of its two parts is below $\varepsilon$. We can estimate each part by saying they are below $\frac{\varepsilon}{2}$:
\[
|f(x) - M| < \frac{\varepsilon}{2}
\]
\[
|g(x) - N| < \frac{\varepsilon}{2}
\]
So let $\delta_1$ and $\delta_2$ both equal to $\frac{\varepsilon}{2}$.


Now choose $\delta = \min(\delta_1, \delta_2)$:
\[
\begin{aligned}
| (f(x) - g(x)) - (M - N) | 
&= | (f(x) - M) - (g(x) - N) | \\
&\leq |f(x) - M| + |g(x) - N| \\
&< \frac{\varepsilon}{2} + \frac{\varepsilon}{2} = \varepsilon.
\end{aligned}
\]

Hence, for every $\varepsilon > 0$, there exists a $\delta > 0$ such that if $0 < |x - c| < \delta$, then $|f(x) - g(x) - (M - N)| < \varepsilon$. Thus, we can conclude:
\[
\lim_{x \to c} [f(x) - g(x)] = M - N.
\]













Hence, for every $\varepsilon > 0$, there exists a $\delta > 0$ such that if $0 < |x - c| < \delta$, then 
$|f(x) - L| < \varepsilon$. Thus, we can conclude:
\[
\lim_{x \to c} [f(x) - g(x)] = M - N
\]



\end{proof}

\end{document}